\documentclass{article}
\usepackage[a4paper]{geometry}
\usepackage{listings}
\usepackage{xcolor}

\title{Digichess C code style guide}
\author{Big Chungus}

\lstset{
	language=C,
	backgroundcolor=\color[HTML]{EFEFEF},
	keywordstyle={\bfseries \color{blue}},
	commentstyle={\color{gray}},
	tabsize=4
}

\begin{document}
\maketitle

\section{Indentation}

\begin{itemize}
	\item Use tabs or else.
	\item Indent cases in switch statements
\end{itemize}

\section{Naming}

\begin{itemize}
	\item Constants in \lstinline{UPPER_SNAKE_CASE}
	\item Local variables in \lstinline{lower_snake_case}
	\item Function names in \lstinline{lower_snake_case}
	\item Typedef struct names in \lstinline{PascalCase}
	\item Don't use globals
\end{itemize}

\section{Headers}

\begin{itemize}
	\item Header file names are of the format \lstinline{lower_snake_case.h}
	\item Include guard macros of the format \lstinline{__UPPER_SNAKE_CASE_H}
	\item For domain-specific code, e.g. a driver, use a prefix for functions and types, e.g.
	
	\begin{lstlisting}
		// Bad
		Point a;
		draw_pixel(a);

		// Good
		LCD_Point a;
		lcd_draw_pixel(a);
	\end{lstlisting}
\end{itemize}

\section{Comments}

\begin{itemize}
	\item Space at the start of a comment, e.g. \lstinline{// Comment here}
	\item Every line in a multiline comment starts with a \lstinline{//}
	\item Doxygen comments use a triple slash, e.g. \lstinline{/// @doxygenhere}
\end{itemize}

\section{Macros}

\begin{itemize}
	\item Always use parentheses around values, e.g. \lstinline{#define CONST_NAME (VALUE)}
	\item If it's a macro used in a limited scope, e.g. just for specific driver header files, use a double underscore, e.g. \lstinline{#define __PRIVATE_MACRO_NAME} 
\end{itemize}

\section{Variables}

\begin{itemize}
	\item Multiple same-type variable declarations in a single line are fine, but don't do single-line multi-variable assignments, e.g.
	\begin{lstlisting}
		int a, b, c; // Fine
		int a = 1, b = 2, c = 3; // No
	\end{lstlisting}

	\item Pointer asterisk next to type, e.g. \lstinline{int* ptr}
	\item Don't do multi-variable single-line pointer assignments.
	\begin{lstlisting}
		int *a, *b; // Bad

		// Good
		int* a;
		int* b;
	\end{lstlisting}
\end{itemize}

\section{Flow control}

\begin{itemize}
	\item Spaces around keywords and same-line braces, e.g. \lstinline{if (condition) { }
	\item New lines around blocks, e.g.
	\begin{lstlisting}
		// Block 1
		if (cond1) {
			foo();
		}

		// Block 2
		if (cond2) {
			bar();
		}
	\end{lstlisting}

	\item Else and else if statements start on the same line as the brace, e.g.
	\begin{lstlisting}
		if (cond1) {
			a();
		} else if (cond2) {
			b();
		} else {
			c();
		}
	\end{lstlisting}

	\item Single-line \lstinline{if} statements without braces are fine if they're kept short, e.g.
	
	\begin{lstlisting}
		// Bad
		int foo(int a) {
			if (a == 5) return a_long_and_complex_function_name_or_operation(a);

			return 0;
		}

		// Better
		int foo(int a) {
			if (a == 5) {
				return a_long_and_complex_function_name_or_operation(a);
			}

			return 0;
		}

		// Good
		int bar(int a) {
			if (a == 5) return 1;

			return 0;
		}
	\end{lstlisting}

	\item In all other cases avoid single-line \lstinline{if} statement and split them into a new line with braces.
\end{itemize}

\section{Functions}
\begin{itemize}
	\item Use return guards when possible, e.g.
	\begin{lstlisting}
		void some_func(int a, int b) {
			if (a == 0) return;
			if (b == 5) return;

			return a + b;
		}
	\end{lstlisting}

	\item For the love of god don't use \lstinline{goto}
	\item Wrap short inline assembly in their own functions. Longer assembly should go into .S files.
\end{itemize}

\section{Operators}

\begin{itemize}
	\item Spaces around operators, e.g. \lstinline{5 + 3}
	\item Always use parentheses around bitwise operations, e.g. \lstinline{2 + (5 << 3)}
	\item If splitting a long chain of operators into multiple lines, put the operator the line is split around in the new line, e.g.
	\begin{lstlisting}
		if (
			arg1
			|| arg2
			|| arg3
		) {
			int value =
				123
				+ 456
				+ 789;

			stuff();
		}
	\end{lstlisting}
\end{itemize}

\section{Misc}

\begin{itemize}
	\item Try to keep the line width under 80 characters
	\item For stuff that's not listed, use your best judgement
\end{itemize}

\end{document}